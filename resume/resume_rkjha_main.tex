\documentclass[9pt]{article}
\usepackage{fullpage}
\usepackage{amsmath}
\usepackage{amssymb}
\usepackage{url}
\usepackage{multicol}
\usepackage[usenames]{color}
\usepackage{enumitem}
\usepackage{nopageno}

\renewcommand{\arraystretch}{1.5}
\setlist{nolistsep}
\leftmargin=0.25in
\oddsidemargin=0.25in
\textwidth=6.0in
\topmargin=-0.75in
\textheight=10.50in

\raggedright
\pagestyle{empty}
%\pagenumbering{arabic}

\def\bull{\vrule height 0.8ex width .7ex depth -.1ex }

\newenvironment{changemargin}[2]{%
  \begin{list}{}{%
    \setlength{\topsep}{0pt}%
    \setlength{\leftmargin}{#1}%
    \setlength{\rightmargin}{#2}%
    \setlength{\listparindent}{\parindent}%
    \setlength{\itemindent}{\parindent}%
    \setlength{\parsep}{\parskip}%
  }%
  \item[]}{\end{list}
}

\newcommand{\lineover}{
	\begin{changemargin}{-0.05in}{-0.10in}
		\vspace*{-8pt}
		\hrulefill \\
		\vspace*{-2pt}
	\end{changemargin}
}

\newcommand{\header}[1]{
	\begin{changemargin}{-0.75in}{-0.75in}
		\scshape{#1}\\
  	\lineover
	\end{changemargin}
}

\newcommand{\name}[1]{
	\begin{changemargin}{-0.6in}{-0.6in}
		\begin{center}
			{\Large \scshape {#1}}
		\end{center}
	\end{changemargin}
}

\newcommand{\contact}[6]{
	\begin{changemargin}{-0.65in}{-0.65in}
		\begin{multicols}{2}
			{#1}\\ \smallskip 
			{#2}\\ \smallskip
			{#3}\\ \smallskip
			{#4}\\ \smallskip 
			{#5}\\ \smallskip
			{#6}
		\end{multicols}
	\end{changemargin}
}

\newenvironment{body} {
	\vspace*{-16pt}
	\begin{changemargin}{-0.6in}{-0.65in}
  }	
	{\end{changemargin}
}	

\newcommand{\school}[4]{
	\textbf{#1} \hfill \emph{#2\\}
	#3\\ 
	#4\\
}

% END RESUME DEFINITIONS

\begin{document}

%%%%%%%%%%%%%%%%%%%%%%%%%%%%%%%%%%%%%%%%%%%%%%%%%%%%%%%%%%%%%%%%%%%%%%%%%%%%%%%%
% Name
\name{Rohit Kumar Jha}

\contact{Junior Undergraduate}{Department of Computer Science and Engineering}{Indian Institute of Technology, Kanpur}{\hspace{60pt} \url{http://home.iitk.ac.in/~rkjha}}{\hspace{60pt} \textbf{Email:} rkjha.iitk@gmail.com}{\hspace{60pt} \textbf{Phone No.:} +91-8005451623}

%%%%%%%%%%%%%%%%%%%%%%%%%%%%%%%%%%%%%%%%%%%%%%%%%%%%%%%%%%%%%%%%%%%%%%%%%%%%%%%%
% Education
\header{Education}

\vspace{6pt}
	\begin{tabular}{ | l | l | l | l | }
	\hline	
	\textbf{Year} & \textbf{Degree/Certificate} & \textbf{Institute} & \textbf{CGPA/Percentage} \\ \hline
	2015 (expected) & B.Tech. & Indian Institute of Technology, Kanpur & 9.6/10 (4 semesters) \\ \hline
	2011 & AISSCE, XII (CBSE) & Delhi Public School, Bokaro Steel City & 92.2\% \\ \hline
	2009 & AISSE, X (CBSE) & Rose Public School, Darbhanga & 96.2\% \\ \hline
	\end{tabular}

\smallskip

%%%%%%%%%%%%%%%%%%%%%%%%%%%%%%%%%%%%%%%%%%%%%%%%%%%%%%%%%%%%%%%%%%%%%%%%%%%%%%%%
% Interests
\header{Areas of Interest}

\begin{body}
	\vspace{14pt}
	\begin{itemize}
		\item{\textbf{Algorithms}}
		\item{\textbf{Machine Learning}}
		\item{\textbf{Natural Language Processing}}
	\end{itemize}
\end{body}

%%%%%%%%%%%%%%%%%%%%%%%%%%%%%%%%%%%%%%%%%%%%%%%%%%%%%%%%%%%%%%%%%%%%%%%%%%%%%%%%
% Academic Achievements
\header{Scholastic Achievements}

\begin{body}
	\vspace{14pt}
	\begin{itemize}
	\item{Awarded \textbf{Gold Medal} for being \textbf{Regional Topper} in \textbf{Regional Mathematical Olympiad (RMO)} 2010 organised by \emph{Homi Bhabha Centre for Science Education}}
	\item{Secured \textbf{All India Rank(AIR) 122} in \textbf{IIT-JEE 2011} out of around \textbf{0.5 million students}}
	\item{Secured \textbf{All India Rank(AIR) 82} in \textbf{AIEEE 2011} out of around \textbf{1.1 million students}}
	\item{Received \textbf{Academic Excellence Award} that is awarded to \textbf{top 7\% students} in Academic Year 2011-12}
	\item{Recipient of \textbf{Summer Undergraduate Research Grant for Excellence (SURGE) 2013}, granted by \emph{Dean Resource Planning and Generation}, IIT Kanpur}
	\item{Secured \textbf{International Rank - 39} in \textbf{4th International Mathematics Olympiad} organised by \emph{Science Olympiad Foundation(SOF)}}
	\item{Secured \textbf{All India Rank(AIR) - 47} in \textbf{11th National Science Olympiad(NSO)} organised by \emph{Science Olympiad Foundation(SOF)}}
	\item{Secured \textbf{All India Rank(AIR) - 12} in \textbf{National Science Talent Search Examination(NSTSE) 2011}}
	\end{itemize}
\end{body}

%%%%%%%%%%%%%%%%%%%%%%%%%%%%%%%%%%%%%%%%%%%%%%%%%%%%%%%%%%%%%%%%%%%%%%%%%%%%%%%%
% Major Projects and Works
\header{Major Projects}

\begin{body}
	\vspace{14pt}
	\textbf{Implementation of Integer Multiplication} \hfill \emph{Summers '13}\\
	\emph{Research Project under SURGE program under the mentorship of Prof. Piyush P. Kurur}
	\begin{itemize}
		\item{Implemented \textbf{Integer Multiplication Algorithms based on Fast Fourier Transformation} especially for 1000 to 10000 bit multiplications due to wide application of 1024, 2048 and 4096 bit multiplications in Cryptanalysis}
		\item{Achieved around \textbf{two times faster running time than the existing GMP Implementation}}
		\item{Planning to get it included in GMP Library}
	\end{itemize}
	\smallskip
	\textbf{Study of Graph Algorithms} \hfill \emph{January '13 till now}\\
	\emph{Research Project under Prof. Surender Baswana}
	\begin{itemize}
		\item{Designed some \textbf{promising ideas} for \textbf{Decremental Maintenance of DFS Tree in Directed Acyclic Graphs} and currently working on them}
		\item{Surveyed the existing literature on \textbf{Incremental Maintenance of All Pairs Reachability in Directed Acyclic Graphs}}
		\item{Studied efficient algorithms for \textbf{Incremental Maintenance of Topological Ordering in Directed Acyclic Graphs} and \textbf{Incremental Maintenance of DFS Tree in Directed and Undirected Graphs}}
	\end{itemize}
	\smallskip
	\textbf{VANI - Institute Wiki for IIT Kanpur} \hfill \emph{May '12 to October '12}\\
	\emph{Supported by Dean, Resource Planning and Generation and Student's Gymkhana, IIT Kanpur}
	\begin{itemize}
		\item{Worked in a team of around 10 to develop an \textbf{open-to-edit wiki built using Drupal CMS} for IIT Kanpur residents where all archival and other important campus information will be available to them}
		\item{Extended to other features like \textbf{Student and Faculty Search},  \textbf{Lost and Found Portal}}
		\item{Included a \textbf{Forum for discussion} with each person getting a platform to opine his/her views by means of a minimalistic profile}
	\end{itemize}
	\smallskip
	\textbf{Hit Me - 3D Game Integrating Windows Phone with Microsoft Kinect} \hfill \emph{March '13 to Apr '13}\\
	\emph{Independent Project}\\
	\begin{itemize}
		\item{Created a 3D game, similar to Temple Run, which more than one player can play}
		\item{\textbf{Integrated Windows Phone and Microsoft Kinect} for a better gaming experience, allowing anyone with a Windows Phone to connect to the game in real time and create obstacles}
		\item{Won \textbf{first position in Softkriti, Techkriti '13}}
		\item{Among \textbf{top 7 teams in Code.Fun.Do organised by Microsoft} and advanced to the \textbf{National Round}}
	\end{itemize}
	\smallskip
	\textbf{8-bit General Purpose Computer} \hfill \emph{March '13 to April '13}\\
	\emph{CS220 Course Project under the guidance of Prof. Subhajit Roy}
	\begin{itemize}
		\item{Implemented a \textbf{8-bit General Purpose Computer on FPGA board} having \textbf{load-store architecture} using Verilog HDL}
		\item{Implemented \textbf{Recursive function calls} and the General Purpose Computer could run programs such as finding factorial of any given number n, finding the sum upto n numbers, finding the nth Fibonacci number and display blinking LED}
		\item{Implemented an \textbf{Interpreted and Compiler} for the programs to be run on this General Purpose Computer}
		\item{Chosen as \textbf{one of the best projects} in the course}
	\end{itemize}
	\smallskip
	\textbf{eC - Higher Level Language for C Language} \hfill \emph{May '12 to June '12}\\
	\emph{Under Programming Club, IIT Kanpur}
	\begin{itemize}
		\item{Designed a much \textbf{easy-to-code-in language}, along with its interpreter, to produce its equivalent C code}
		\item{Provides a \textbf{huge relaxation in terms of syntax} and provides \textbf{functions for most common tasks}}
		\item{\textbf{Takes care of the common syntax errors} that tend to creep in}
	\end{itemize}
	\smallskip
	\textbf{Cloud Pad} \hfill \emph{Summers '12}\\
	\emph{Under Electronics Club, IIT Kanpur}
	\begin{itemize}
		\item{Developed a \textbf{GUI for PandaBoard} containing a game, bluetooth application and a music and video player}
		\item{Game contains a \textbf{Multiplayer LAN Game and a Cloud-based Game} for PandaBoard}
		\item{Bluetooth Application allows \textbf{Realtime Maintenance and Access of Information on PandaBoard} using Bluetooth with Information residing at Remote Server}
	\end{itemize}
	\smallskip
	\textbf{Hexapod} \hfill \emph{Summers '12}\\
	\emph{Under Robotics Club, IIT Kanpur}
	\begin{itemize}
		\item{Built a six-legged desktop sized Hexapod, with three degrees of freedom in each leg and independent control over each leg}
		\item{Excellent design and walking mechanism renders stability in spite of reasonable weight}
		\item{Can perform all kinds of rotation with great precisions, and can walk in all directions}
	\end{itemize}
	\smallskip
	\textbf{Table Tennis Game} \hfill \emph{April '13}\\
	\emph{CS252 Course Project under the guidance of Prof. Arnab Bhattacharya}\\
	\smallskip
	\begin{itemize}
		\item{Created a 2D Table Tennis Game in Python, both for two player and with an AI, containing sound effects and other features using PyGame module}
	\end{itemize}
	\smallskip
	\textbf{Snake Game} \hfill \emph{November '11}\\
	\emph{ESC101 Course Project under the guidance of Prof. Piyush P. Kurur}\\
	\begin{itemize}
		\item{Created a Snake Game in Python with some innovative and interesting features using PyGame module}
		\item{Got \textbf{A* in the course based on the project work} and for being in top 1\% in the course}
	\end{itemize}
\end{body}

%%%%%%%%%%%%%%%%%%%%%%%%%%%%%%%%%%%%%%%%%%%%%%%%%%%%%%%%%%%%%%%%%%%%%%%%%%%%%%%%
% Relevant Courses
\header{Relevant Courses (*Ongoing)}

\begin{body}
	\vspace{4pt}
	\begin{multicols}{2}
	Fundamentals of Computing\\
	Mathematics I (Basic Calculus)\\	
	Mathematics II (Linear Algebra)\\
	Data Structures and Algorithms\\ 
	Computer Organization\\
	Mathematics for Computer Science I\\	
	Mathematics for Computer Science II\\	
	Mathematics for Computer Science III\\	
	Computing Lab I\\
	Computing Lab II\\
	Probability and Statistics\\
	Introduction to Logic\\
	*Randomized Algorithms\\
	*Natural Language Processing\\
	*Operating Systems\\
	*Theory of Computation\\
	*Modern Cryptology
	
	\end{multicols}
\end{body}

%%%%%%%%%%%%%%%%%%%%%%%%%%%%%%%%%%%%%%%%%%%%%%%%%%%%%%%%%%%%%%%%%%%%%%%%%%%%%%%%
% Skills
\header{Computer Skills}

\begin{body}
	\vspace{14pt}
	\emph{\textbf{Languages (Proficient in):}}{} C, C++, Python\\
	\emph{\textbf{Languages (Familiar with):}}{} C\#, Java, Javascript, PHP, \\
	\emph{\textbf{Other Tools/Languages:}}{}  \LaTeX , HTML, CSS, Octave, Unity 3D, Bash Shell Scripting, Drupal(CMS), Git, MIPS Assembly Language, Verilog\\
	\emph{\textbf{Platforms:}}{} Windows, Linux, Windows Phone, Android
\end{body}

%%%%%%%%%%%%%%%%%%%%%%%%%%%%%%%%%%%%%%%%%%%%%%%%%%%%%%%%%%%%%%%%%%%%%%%%%%%%%%%%
% Other Achievements
\header{Other Achievements}

\begin{body}
	\vspace{14pt}
	\begin{itemize}
		\item{Won \textbf{AI Challenge} organized by \textbf{HackerRank} for IIT Kanpur Students}
		\item{Achieved \textbf{1st Position} in \textbf{FPGA Contest} at \textbf{Techkriti}, \emph{the Annual Tech Festival of IIT Kanpur}}
		\item{Achieved \textbf{1st Position} in \textbf{Softkriti}, the \emph{Software Innovation Challenge}, at \textbf{Techkriti}, the \emph{Annual Tech Festival of IIT Kanpur}}
		\item{Advanced to Second Round, \textbf{the National Round}, of \textbf{Code.Fun.Do 2013} organised by \textbf{Microsoft}}
		\item{Received \textbf{Hacker's Choice Award} in \textbf{Yahoo HackU 2012}}
	\end{itemize}
\end{body}

\end{document}
