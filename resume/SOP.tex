
\documentclass{article}
% \usepackage[T1]{fontenc}
% \usepackage[utf8]{inputenc}
\usepackage[margin=1in]{geometry}
% \usepackage{asm}
\newcommand{\HRule}{\rule{\linewidth}{0.5mm}}
\newcommand{\Hrule}{\rule{\linewidth}{0.3mm}}

\makeatletter% since there's an at-sign (@) in the command name
\renewcommand{\@maketitle}{%
  \parindent=0pt% don't indent paragraphs in the title block
  \centering
  {\Large \bfseries\textsc{\@title}}
  \par
  \textit{\@date}
  \HRule\par%
  \textit{\@author}
  % \par
  % \textit{\@author}
  \par
}
\makeatother% resets the meaning of the at-sign (@)

\title{Statement of Purpose}
\author{Siddhant Manocha \hfill CPI: 9.9/10.0  \\ Department of Computer Science and Engineering \hfill IIT Kanpur  }
% \email{9.9/10}
\date{Cornell Summer Research Internship Application}

\begin{document}
  \maketitle% prints the title block
  \vspace{0.5cm}
I am a dedicated and hard working student with keen interest in practical application of Computer Science. I have been enthusiastic about the possibilities of Computer Science and have been  exploring various areas in this field from the very start of my undergraduate study. I developed interest in web development during initial years at college and undertook projects and participated in various competitions. I also learnt about software engineering, machine learning, computer vision, and computer networks through various courses and research projects. With my experience at research so far, I am interested in the field of Machine Learning and Computer Vision. \\

In my second year at college, I started to watch lectures of machine learning course on Coursera and learnt about various methods and techniques. Subsequently, I got involved in a semester project on manifold learning under Dr. Amitabha Mukerjee. I attended various presentations by his graduate students which were mainly in the field of machine learning and learnt a lot of new exciting approaches and applications. I studied various methods for dimensionality reduction like PCA, Isomap, Locally Linear Embedding ,etc. \\

I was selected Summer Undergraduate Research Grant for Excellence program here at IIT Kanpur, through which I got the opportunity to pursue a research project under Dr. Amitabha Mukerjee. The project aimed at formulating a solution to automate the motion and plan the trajectory of CRS Robotic arm (having 3 degrees of freedom) using computer vision, image processing and gaussian models. I studied various approaches and utilised Gaussian Process Latent variable model and Gaussian process regression to devise a solution to the problem. My research experience was very fruitful and I got hands on experience in learning about the practical utilities of Machine Learning and Computer Vision. \\

The second research experience I had with Machine Learning was at the Carnegie Mellon Winter School held at  NIT Karnataka where I worked on the project “Automatic Commentary Generation for Tennis Matches” under the guidance of Prof.Bhiskha Raj, CMU and Pulkit Agarwal, Phd, UC Berkeley last December. We divided the problem into separate components which included tennis court detection, object recognition, action classification and tennis ball tracking to extract useful information from videos. Our group implemented Kernel SVM on HOF features (Histogram of Optical Flow) for action recognition and Optical Flow for tennis ball tracking. We also used RCNN toolbox for object detection in tennis matches. We were successful in formulating a solution for individual components and were awarded one of the \"Best Projects\" at the Winter School. We are still working on natural language generation from the output obtained from various components and make an end to end system. \\

% I have really enjoyed learning from my previous projects which are related to practical applications of various concepts. I find the field of research very exciting and plan to pursue research as my career. Internship at Cornell University provides me with a platform to experience world class research environment and pursue projects related to practical applications of Computer Science. The research experience at Cornell University, which is highly reputed for its research facilities, will give a unique and incomparable experience to me. I will work hard to get valuable research experience which would be helpful at later stage in my career. At the end of my internship, I hope to be in a better position to choose an area of research for my graduate studies and  return with a better personality, professionalism and a broader perspective on research. I hope that my background and qualifications are found suitable for the this internship. \\

I have really enjoyed learning from my previous projects which are related to practical applications of various concepts. I find the field of research very exciting and plan to pursue research as my career. 
An opportunity to work as a part of RoboBrain Project provides me with a excellent platform to pursue projects related to practical applications of Machine Learning and Computer Vision. Moreover, the research experience at Cornell University, which is highly reputed for its research facilities, will give a unique and incomparable experience to me. I will work hard to get valuable research experience which would be helpful at later stage in my career. At the end of my internship, I hope to be in a better position to choose an area of research for my graduate studies and return with a better personality, professionalism and a broader perspective on research. I hope that my background and qualifications are found suitable for the this internship. \\
\end{document}
